
\newpage

\subsection*{Orthogonal Least Squares}
In the last lecture we have introduced the least squares method. In particular we have mentioned the linear model for which:

$$
\underline{\hat{y}}=X \underline{\hat{w}} \quad X \in \mathbb{R}^{n \times p} \quad \underline{\hat{y}} \in \mathbb{R}^{n} \quad n \geq p \quad X \text { full rank }
$$

We have obtained:

$$
\underline{\hat{w}}=\left(X^{\top} X\right)^{-1} X^{\top} \underline{y} \Longrightarrow \underline{\hat{y}}=\underbrace{X\left(X^{\top} X\right)^{-1} X^{\top}}_{P_{x}} \underline{y}
$$

The matrix $P_{x}$ dimension is given by the product of: $(n \times p)(p \times p)(p \times n)=(n \times n)$ and has this properties:
- $P_{x}=P_{x}^{2}$
- $P_{x}$ is a projection matrix

Let's consider $U$ an orthogonal $\left(U^{\top} U=I\right)$ matrix that contains the basis for $\mathcal{C}(X)$ this means that $\mathcal{C}(X)=\mathcal{C}(U)$. We can write:

$$
\underline{\hat{y}}=X \underline{\hat{w}}=U \underline{\tilde{w}}
$$

So this basically means that the predicted value of $y$ still a projection on a plane but this time the plane is spanned by the columns of $U$ and not by the columns of $X$. By substituting last equation in the minimization method for least squares we have:

$$
\underline{\tilde{w}}=\underset{\underline{w}}{\arg \min }\|\underline{y}-U \underline{w}\|_{2}^{2} \Longrightarrow \underline{\hat{y}}=U \underline{\tilde{w}}=U\left(U^{\top} U\right)^{-1} U^{\top} \underline{y}=U U^{\top} \underline{y}
$$

This formulation is possible because this time in the parenthesis we have an orthogonal matrix and this means that $\left(U^{\top} U\right)^{-1}=U^{\top} U=I$. In general $U U^{\top} \neq I$ because it might be rectangular (while $U^{\top} U$ is always square).\\ 
How do we build the orthogonal matrix $U$ ? We can use the Gram-Schmidt procedure, but a drawback of Gram Schmidt is that, depending on the order chosen for the columns of $X$, the matrix $U$ can be different. Moreover, the order of vector columns of $U$ is meaningless.\\
Now, we want to exploit the SVD for computing the orthogonal matrix $U$. We start from the SVD of $X$ :


$$
\begin{aligned}
\underline{\hat{w}} & =\left(X^{\top} X\right)^{-1} X^{\top} y \\
& =\left(V \Sigma^{\top} U^{\top} U \Sigma V^{\top}\right)^{-1} V \Sigma^{\top} U^{\top} \underline{y} \\
& =\left(V \Sigma^{\top} \Sigma V^{\top}\right)^{-1} V \Sigma^{\top} U^{\top} \underline{y} \\
& =V\left(\Sigma^{\top} \Sigma\right)^{-1} \underbrace{V^{\top} V}_{I} \Sigma^{\top} U^{\top} \underline{y} \\
& = V\underbrace{\left(\Sigma^{\top} \Sigma\right)^{-1} \Sigma^{\top}}_{\Sigma^{+}} U^{\top} \underline{y} \\
& = V\Sigma^{+} U^{\top} \underline{y} \\
& = X^{+}\underline{y}
\end{aligned}
$$


$\Sigma^{+}$il called the pseudo-inverse of $\Sigma$.

Eventually, we have:


\\
Let's consider now the case in which \( p \geq n \) and \( X \) has \( n \) linearly independent rows (before we had \( p \) linearly independent columns). This means that we have more unknowns than equations and we would find infinite solutions for \(\underline{\hat{w}}\) such that \(\underline{\hat{y}} = X \underline{\hat{w}}\).

The solution found before, \(\underline{\hat{w}} = V \Sigma^{+} U^{\top} \underline{y}\), is still valid, but now \(\Sigma^{+} = \Sigma^{\top} (\Sigma \Sigma^{\top})^{-1}\). This means it has the same shape as before but transposed. \\ \\
Let's consider two vectors: w_\underline{w} and ˆw_\underline{\hat{w}}:

\[
\underline{y} = X \underline{w} = X \underline{\hat{w}}
\]

So they both are solutions of the system of equations.

To prove that the vector \(\underline{\hat{w}} = V \Sigma^{+} U^{\top} \underline{y}\) is the solution with the minimum norm among all solutions \(\underline{w}\) of the system \(\underline{y} = X \underline{w}\), we will use the properties of the SVD and the orthogonality of the matrices \(U\) and \(V\).\\ \\
We have that among the infinite solution, the one obtained through the psudo inverse is the on with minimal norm. \\
\begin{theorem}
For an underdetermined system $\underline{y} = X\underline{w}$, the solution $\hat{w}$ computed through the pseudoinverse has the minimum norm among all solutions.
\end{theorem}

\begin{proof}
Let $w$ be any solution to the underdetermined system, and $\hat{w}$ be the solution computed through the pseudoinverse. We will show that $\|\hat{w}\|_2^2 \leq \|w\|_2^2$.

We begin by expanding $\|w\|_2^2$:

\begin{align*}
\|w\|_2^2 &= \|w-\hat{w}+\hat{w}\|_2^2 \\
&= (w-\hat{w}+\hat{w})^{\top}(w-\hat{w}+\hat{w}) \\
&= (w-\hat{w})^{\top}(w-\hat{w}) + \hat{w}^{\top}\hat{w}li + 2(w-\hat{w})^{\top}\hat{w}
\end{align*}

Now, we use the following properties:

\begin{enumerate}
\item Both $w$ and $\hat{w}$ are solutions, so $Xw = X\hat{w} = y$. This implies $X(w - \hat{w}) = 0$, meaning $(w - \hat{w})$ is in the null space of $X$.\\

\item $\hat{w}$, being the pseudoinverse solution, is orthogonal to the null space of $X$. Therefore, $(w - \hat{w})^{\top}\hat{w} = 0$.\\
\end{enumerate}

Using these properties, we can simplify our expansion:

\begin{align*}
\|w\|_2^2 &= (w-\hat{w})^{\top}(w-\hat{w}) + \hat{w}^{\top}w + 2(w-\hat{w})^{\top}\hat{w} \\
&= (w-\hat{w})^{\top}(w-\hat{w}) + \hat{w}^{\top}\hat{w} + 0 \\
&= \|w-\hat{w}\|_2^2 + \|\hat{w}\|_2^2
\end{align*}

Since $\|w-\hat{w}\|_2^2$ is always non-negative, we can conclude that:

\begin{equation*}
\|w\|_2^2 \geq \|\hat{w}\|_2^2
\end{equation*}

This inequality holds for any solution $w$. Therefore, $\hat{w}$ has the minimum norm among all solutions of the system $\underline{y} = X\underline{w}$.
\end{proof}

\newpage

\section{Matrix completion}
We start from 
\[
    X \in \mathbb{R}^{n \times p}  \hspace{0.5cm} rank(X) = r \ll \min(n,p) 
\]
We know only partially X, we know $X_{i,j} \texttt{for} (i,j)\in\Omega(i,j) \in \Omega.$ 

Matrix completion is a method for filling missing values. If we had full rank matrix we would have independent columns so we would not be able to retrieve/obtain missing values. Being low rank, in this sense, helps us accomplishing this goal.  

\subsection*{Ideal estimator}
\[
    \begin{cases}
        \hat{X} = \underset{z \in \mathbb{R}^{n \times p}}{\arg \min} \left[\text{ rank}(z)\right]\\
        \text{subject to } \hat{X_{ij}} = X_{ij} (i,j) \in \Omega
    \end{cases}    
\]
This formulation is computationally unfeasible as  the object function is non-convex. Amongst all possible solutions, we are searching for the one with minimal rank.
The problem is that this is a non-convex optimization problem which requires a lot of computational power to be solved.

\subsection*{Practical estimator}
We are going to use the \textbf{Nuclear norm}: $||\cdot||_*  = \sum\limits_{i=1}^{\min(n,p)}\sigma_i$.

The idea is that, instead of minimizing the rank, we minimize the norm since, when performing the SVD, the number of non-zero singular values is equal to the rank of the matrix.
\[
    \begin{cases}
        \hat{X} = \underset{z \in \mathbb{R}^{n \times p}}{\arg \min} ||z||_*\\
        \text{subject to } \hat{X_{ij}} = X_{ij} (i,j) \in \Omega
    \end{cases}    
\]
This new formulation is convex optimization problem. How are we going to solve this problem? With the \textbf{SVT (Singular Value Threshold)}. Here is the algorithm:
\begin{itemize}
    \item Initialize $\hat{X} = \texttt{zeros}(n,p)$
    \item Set $\hat{X}_{ij} = X_{ij} \texttt{ for } (i,j) \in \Omega$
    \item For $k = 1,2, \dots, N$  
    \begin{itemize}
        \item $\hat{X}_{old} = \hat{X}$
        \item $[U, \Sigma, V^\intercal] = \texttt{svd}(\hat{X})$
        \item $\Sigma \rightarrow \hat{\Sigma} \hspace{0.2cm} \begin{cases}
            \hat{\sigma}_i = \sigma_i & \text{ if } \sigma_i > \tau\\
            \hat{\sigma}_i = 0 & \text{ otherwise }
        \end{cases}$
        \item $\hat{X} = U\hat{\Sigma}V^\intercal$
        \item $\hat{X}_{ij} = X_{ij} \texttt{ for } (i,j) \in \Omega$
    \end{itemize}
    \item $||\hat{X} - X_{old}|| < \epsilon$
\end{itemize}
The constant $\tau$ is imposing some sort of thresholding on the acceptance of singular values. This is an example of non-monothone algorithm because after the reduction of rank with the condition on sigmas, we override some value of the matrix to the one contained in $\Omega$. Is it proved that for a large enough index $k$ the algorithm converge to the solution. 